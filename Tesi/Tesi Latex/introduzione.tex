\chapter*{Introduzione}

% \textcolor{red}{METTEREI UNA FRASE CHE INTRODUCA IL TEMA CON UNO SGUARDO PIù AMPIO. QUI UNA PROPOSTA MA PUò AMPLIARE A PIACERE\\
% Prevenire gli scarti di produzione è uno dei principali obiettivi in tutti processi produttivi. 
% }

La prevenzione degli scarti di produzione gioca un ruolo molto importante per il miglioramento del rendimento e delle condizioni finanziarie di qualsiasi azienda. \\
In effetti, l'eliminazione dei difetti ha un effetto diretto sul margine di profitto del prodotto e diminuisce il costo della qualità durante la fabbricazione del prodotto. 
Le aziende si sforzano di ridurre il più possibile il tasso di difetti del prodotto mediante un costante controllo ed ispezione in diversi punti del ciclo di produzione. 
\cite{herdt2010maternal} 


La presente tesi concerne l’analisi del processo relativo alla produzione di innerliner, quale componente che riveste lo pneumatico. \\
Il fine del lavoro è lo studio di come prevenire gli scarti di produzione interni allo stabilimento della Pirelli di Settimo Torinese.
La metodologia utilizzata è stata quella di monitorare costantemente la qualità del prodotto; conseguente a un'accurata comprensione della politica aziendale di gestione dei prodotti non conformi, ossia dei prodotti che non rispettano le specifiche richieste. \\
La prima fase della ricerca si è composta da un'accurata analisi di una serie di campioni relativi alla produzione di innerlinner.
Successivamente, attraverso un identificazione delle cause di variabilità ed il controllo degli indici di processo $CP$ e $CPK$, si è cercato di comprendere come ottenere dei prodotti conformi alle specifiche tecniche stabilite nel piano di controllo. \\
Lo scopo aziendale è quello di assicurare che il processo di produzione e di controllo sia in grado di garantire la conformità del prodotto. 
Ci si riferisce, quindi, sia ai prodotti finiti (lo pneumatico) e sia ai prodotti definiti semilavorati.
A tal proposito sono stati monitorati i processi interni di qualità nella gestione di scarti e scarti di prodotto finito. 
 

L'obiettivo di questa analisi è analizzare come sarebbe possibile migliorare le prestazioni della produzione di innerliner attraverso un costante utilizzo del controllo statistico di processo: utilizzando misure online, estraendo i dati dallo strumento ottico e definendo un appropriato algoritmo di filtraggio e analisi dei dati per la valutazione del prodotto e del processo.

Al fine di rispondere agli obiettivi illistrati, la presente tesi si articola in tre capitoli. \\
Il primo capitolo, ``Review della letteratura'' è un indagine sulle fonti scientifiche relative al miglioramento di un processo produttivo. \\
Il secondo capitolo, ``La metodologia: dai Big Data all'analisi della capacità di un processo'' da una prima definizione ai ``Big Data'' per poi illustrarne i principali aspetti positivi e negativi. Nella seconda parte di questo capitolo viene illustrata la metodologia statistica che si adopererà nell'elaborato: l'analisi di processo, lo studio della variabiltà e gli indici per misurare la variabilità di un processo. \\
Il terzo capitolo concerne l'analisi empirica rivolta allo studio della produzione di innerliner, questo capitolo si articola in tre step principali: introduzione dell'azienda ed illustrazione del ciclo produttivo di uno pneumatico, strumenti utilizzati e linee guida da seguire per la costruzione di un'analisi di processo; validazione dei dati, analisi esplorativa, costruzione ed elaborazione dell'analisi di processo.

Grazie all’analisi del caso di studio, dopo aver seguito i passaggi descritti nel terzo capitolo, si è compreso il comportamento del processo di produzione di innerliner, quindi, sono state identificate alcune fonti di variabiltà che se, monitorate costantemente potrebbero aiutare nell'obiettivo della prevenzione dei difetti. 

% \textcolor{red}{FORSE DIREI QUALCOSA DI SPECIFICO SULLA METODOLOGIA
% E POI RACCONTEREI LA STRUTTURA DELLA TESI
% \\Infine, alcune considerazioni finali concludono l'elaborato. }