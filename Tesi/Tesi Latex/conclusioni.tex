\chapter*{Conclusioni}
\label{chap:Conclusioni}
\markboth{Conlusioni}{Conclusioni}



Il fine del lavoro è lo studio di come sarebbe possibile prevenire gli scarti di produzione, ossia dei prodotti che non rispettano le specifiche richieste. 
Tale obiettivo è stato raggiunto anche attraverso l'applicazione di tecniche statistiche, quali l'analisi della capacità di processo. \\
La metodologia utilizzata è stata quella di monitorare la qualità del prodotto, attraverso una prima validazione ed esplorazione dei dati e delle conseguenti distribuzioni dei parametri di interesse.
Successivamente si é proceduto ad estrarre un campione ed analizzare, tramite l'utilizzo di serie storiche, la distribuzione di questo nell'intervallo di produzione. \\
Grazie allo studio di tale serie storica, dopo aver definito un criterio per la rimozione degli outlier, é stato possibile fare una valutazione di screening del processo per uno specifico codice ricetta, identificando i valori anomali che alteravano il campione.
Al fine di calcolare gli indici di capacità del processo $CP$ e $CPK$, si é reso necessario elaborare un test d'ipotesi per verificare la normalità della distribuzione, essendo quest'ultimo un requisito fondamentale. \\
Successivamente é stato possibile costruire gli indici $CP$ e $CPK$ per ogni specifico codice ricetta, ed identificare attraverso quest'ultimi come si comportava il processo produttivo in presenza e o in assenza di un fermo macchina.

Un'intuizione futura, che potrebbe essere interessante esplorare, è legata ai criteri e alle modalità adottate per l'identificazione dei valori anomali, al fine di ridurre la dispersione dei singoli codici ricetta.

La struttura con la quale vengono raccolti i dati dallo strumento ottico consente di condurre l'analisi della capacità di processo per uno specifico codice di ricetta su ogni ciclo di produzione, ma non in base a lotti composti da diverse bobine prodotte con lo stesso codice di ricetta.

Inoltre, una futura implementazione potrebbe essere quella di valutare la qualità del prodotto associata a ciascuna bobina e di definire una procedura per utilizzare attivamente e correttamente lo strumento di misura online, definendo come e dove misurare le caratteristiche di interesse, sulla base del piano di controllo e della conoscenza del processo.

Ancora, in un'ipotesi di riduzione dei difetti, un uso completo e corretto dello strumento di misura potrebbe migliorare la riduzione degli scarti.
In futuro, il progetto che si intende realizzare e che è attualmente in corso, è quello di automatizzare il processo di raccoglimento dei dati e sincronizzare in tempo reale il calcolo degli indici di processo.
La disponibilità di informazioni in tempo reale, tramite un implementazione di una \textit{Dashboard} sfruttando la tecnologia di Data Visualization, troverebbe il suo utilizzo nel identificare e limitare la variabilità del processo e correlare queste informazioni con gli scarti sul prodotto finito. 

In questo modo, l'analisi di capacità sui semilavorati come l'innerliner, potrebbe essere utile per capire le cause profonde dei diversi tipi di difetti e anche la loro relazione con l'uniformità dello pneumatico.


% \textcolor{red}{L'obiettivo della presente tesi era quello di effettuare XXX. Tale obiettivo è stato raggiunto anche attraverso l'applicazione di tecniche statistiche relative all'analisi di capicità di processo. }
% Grazie all'analisi del caso di studio proposto in tale tesi, dopo aver definito un criterio per la rimozione degli outlier, si 
% \textcolor{red}{FORSE AGGIUNGERE QUALCHE CONSIDERAZIONE}